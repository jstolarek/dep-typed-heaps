\section{... (two-pass merge)}

%Before diving into implementetion let's take a quick look at our assumptions. Firstly, I assume that we have definitions of Nat and +. We define priority and rank as nats.

% Maybe explain sized types here? Sized types are used to guide the termination checker, they are not always required, sometimes termination checker can do a good job without them. We encounter some agda problems and sized types sometimes don't work even though they should.

We use following notation:

\begin{itemize}
 \item $h1$, $h2$ - first heap, second heap
 \item $p1$, $p2$ - priority of root element in first/second heap
 \item $l1$, $r1$ - left and right subtrees of first  heap
 \item $l2$, $r2$ - left and right subtrees of second heap
 \item $l1-r$, $r1-r$ - rank of left and right subtrees of first  heap
 \item $l2-r$, $r2-r$ - rank of left and right subtrees of second heap
\end{itemize}


\subsection{No proofs}

We begin with a simple implementation that has no proof of correctness. Let us define \Heap datatype:

\begin{code}
data Heap : Set where
  empty : Heap
  node  : Priority → Rank → Heap → Heap → Heap
\end{code}

\begin{listing}[b!]
\begin{code}
makeT : Priority → Heap → Heap → Heap
makeT p l r with rank l ≥ rank r
makeT p l r | true  = node p (suc (rank l + rank r)) l r
makeT p l r | false = node p (suc (rank l + rank r)) r l

merge : Heap → Heap → Heap
merge empty h2 = h2
merge h1 empty = h1
merge (node p1 h1-r l1 r1) (node p2 h2-r l2 r2)
  with p1 < p2
merge (node p1 h1-r l1 r1) (node p2 h2-r l2 r2)
  | true  = makeT p1 l1 (merge r1 (node p2 h2-r l2 r2))
merge (node p1 h1-r l1 r1) (node p2 h2-r l2 r2)
  | false = makeT p2 l2 (merge (node p1 h1-r l1 r1) r2)
\end{code}
\caption{Implementation of makeT and merge}\label{lst:makeT-merge}
\end{listing}

According to this definition a heap is either empty or it is a node with \Priority, \Rank and two subheaps. Note that storing rank in a node is redundant -- we could just compute size of a tree whenever necessary. The reason why I chose to store \Rank in the constructor is that it will be instructive to show in Section \ref{sec:rank-property} how it is converted into inductive type family index.

\subsubsection{merge}

Heaps are merged using a recursive algorithm. We need to consider four cases:

\begin{enumerate}
 \item (base case) $h1$ is empty - return $h2$.
 \item (base case) $h2$ is empty - return $h1$.
 \item (inductive case) priority $p1$ is higher than $p2$ -- $p1$ becomes new root, $l1$ becomes its one child and result of merging $r1$ with $h2$ becomes the other child.
 \item (inductive case) priority $p2$ is higher than $p1$ -- $p2$ becomes new root, $l2$ becomes its one child and result of merging $r2$ with $h1$ becomes the other child.
\end{enumerate}

There is no guarantee that rank of $r1$ merged with $h2$ (or $r2$ merged with $h1$) will be smaller or equal to rank of $l1$ (or $l2$). To ensure that rank invariant is maintained we use a helper function \makeT, as proposed by Okasaki \cite{Oka99}. We pass new children to \makeT, which creates a new node and swaps the children if necessary (see \Listing{lst:makeT-merge}). As Okasaki points out this algorithm can be view as having two passes: a top-down pass that performs merging and a bottom-up pass that restores the rank invariant.

\subsubsection{insert}

We can now implement merge as described in Section~\ref{sec:wblh}:

\begin{code}
singleton : Priority → Heap
singleton p = node p one empty empty

insert : Priority → Heap → Heap
insert p h = merge (singleton p) h
\end{code}

\subsubsection{findMin}

To retrieve element with the highest priority we return value stored in the root of a heap:

\begin{code}
findMin : Heap → Priority
findMin empty          = \hilight{?}
findMin (node p _ _ _) = p
\end{code}

Here we encounter first serious problem: what should \findMin return for an empty heap? If we were using Haskell one thing we could consider is throwing an error. But Agda is a total language, which means that every function must terminate with a result. Throwing errors is not an option. Another alternative is to assume some default priority that will be returned for an empty heap. This priority would have to be some distinguished natural number. $0$ represents the highest priority so it unreasinable to assume it as default. We could return $\inf$, which represents lower priority than any natural number. But this would require us to extend definition of \Nat with $\inf$ which in turn would force us to modify all functions that pattern match on values of \Nat. Let's face it -- \findMin function is not defined for an empty heap but types don't reflect that! To solve our problem we need to be more specific about our types. One solution would be to use \texttt{Maybe} datatype:

\begin{code}
data Maybe (A : Set) : Set where
  nothing : Maybe A
  just    : A → Maybe A

findMinM : Heap → Maybe Priority
findMinM empty          = nothing
findMinM (node p _ _ _) = just p
\end{code}

Returning \texttt{nothing} is like saying ``no output exists for the given input data''. This allows us to express the fact that \findMin is not defined for some input values. But we can do better by ensuring that \findMin is never called for an empty heap.

\subsubsection{deleteMin}

Definition of \deleteMin is based on description in Section~\ref{sec:wblh}. We face the same problem again:

\begin{code}
deleteMin : Heap → Heap
deleteMin empty          = \hilight{?}
deleteMin (node _ _ l r) = merge l r
\end{code}

\subsection{Prooving rank property}\label{sec:rank-property}

We will now prove that our implementation maintains the rank property. The first step is to express \Rank at the type level as an index of \Heap datatype. Since rank of a heap is now part of its type we can ensure that rank of left subtree is not smaller than rank of the right subtree. We do this be requiring that \node constructor is given a proof that rank invariant holds. To express such proofs we define ≥ datatype:

\begin{code}
data _≥_ : Nat → Nat → Set where
  ge0 : \{  y : Nat\}         → y     ≥ zero
  geS : \{x y : Nat\} → x ≥ y → suc x ≥ suc y
\end{code}
\noindent
Values of this type are indexed by two natural numbers. They prove that: a) any natural number is greater than or equal to \texttt{0} (\texttt{ge0} constructor); b) if two numbers are in greater-equal relation then their successors are also in that relation (\texttt{geS} constructor). This type represents concept of data as evidence~\cite{AltMcBMcK05}. We use \texttt{order} function to compare two natural numbers and \texttt{Order} datatype to express the result. Implementation is located in \texttt{Basics.Ordering} module of the companion code.

Having defined ≥ we can now give new definition of \Heap:

\begin{code}
data Heap : Rank → Set where
  empty : Heap zero
  node  : \{l r : Rank\} → Priority → l ≥ r →
          Heap l → Heap r → Heap (suc (l + r))
\end{code}

\noindent
Empty heap contains no elements and so \Empty returns \Heap indexed with \texttt{0}. Non-empty node stores an element and two children of size \texttt{l} and \texttt{r}. Therefore the size of the resulting heap is \texttt{1} + \texttt{l} + \texttt{r}, which we express as \texttt{suc(l + r)}. We must also supply a value of type \texttt{l ≥ r} to the constructor, ie. we must provide a proof that rank invariant holds.

Proving the rank invariant itself is surprisingly simple. We can easily obtain evidence that rank of left subtree is not smaller than rank of right subtree by using \texttt{order} function, that compares two \texttt{Nat}s and supplies a proof of the result. But there is a another difficulty here. Since we index heaps by their sizes we now require that \makeT and \merge construct trees of correct size. Prooving this requires some substantial work on our side. We need to prove that the size of merged heap is equal to the sum of sizes of heaps being merged. Recall that our merging algorithm is two pass: we use \merge to actually do the merging and \makeT to
restore the rank invariant if necessary. This means our proof will be two-stage. We need to prove that: a) \makeT creates a node of required size; b) recursive calls to \merge produce heaps of required size.

\subsubsection{Proving makeT}

\makeT takes subtrees of rank \texttt{l} and \texttt{r} and produces a new tree with rank \texttt{suc(l + r)}. We must prove that each case of \makeT returns tree of correct size:

\begin{enumerate}
 \item If rank \texttt{l} is greater than or equal to rank \textbf{r} then no extra proof is necessary as everything follows from the definition of +.
 \item If rank \texttt{r} is greater than or equal to rank \textbf{l} then we must swap left and right subtrees. This requires us to prove that:

\begin{equation*}
suc (r + l) ≡ suc (l + r)
\end{equation*}

That proof is done using congruence on suc function and commutativity of addition. We will define that proof as \texttt{makeT-lemma} as we will be using in subsequent proofs.
\end{enumerate}

New code of \makeT is show in \Listing{lst:rank-proof-makeT-two-pass}. Note the use of \texttt{subst}. We use it to apply the proof to the \Heap type constructor and convert the type produced by $(node p r≥l r l)$ expression into type given in the type signature. % not exactly precise - type signature is transformed using definitions.

\begin{listing}[thb!]
\begin{code}
makeT-lemma : (a b : Nat) → suc (a + b) ≡ suc (b + a)
makeT-lemma a b = cong suc (+comm a b)

makeT : \{l r : Rank\} → Priority → Heap l → Heap r → Heap (suc (l + r))
makeT \{l-rank\} \{r-rank\} p l r with order l-rank r-rank
makeT \{l-rank\} \{r-rank\} p l r | ge l≥r
  = node p l≥r l r
makeT \{l-rank\} \{r-rank\} p l r | le r≥l
  = subst Heap (makeT-lemma r-rank l-rank) (node p r≥l r l)
\end{code}
\caption{Implementation of \makeT with verified rank property.}\label{lst:rank-proof-makeT-two-pass}
\end{listing}

\subsubsection{Proving merge}

We need to prove that all four cases of merge -- shown in \Listing{lst:makeT-merge} -- produce heap of required size. To make notation more compact in this section I will use $h1$ and $h2$ to denote ranks of heaps being merged and $l1$, $r1$, $l2$ and $r2$ to denote ranks of their children. $p1$ and $p2$ still denote priorities stored in the roots of merged heaps. The size of merged heap must be equal to the sum of input heaps, ie. $h1 + h2$.

\subsubsection{Proving merge, base cases}

In the first base case we have $h1 ≡ 0$. Therefore:

\begin{equation*}
h1 + h2 ≡ 0 + h2 \stackrel{+, (1)}{≡} h2
\end{equation*}
\noindent
Which ends the first proof -- everything follows from definition of $+$. In the second base case $h2 ≡ 0$ and things are slightly more difficult: the definition of $+$ only says that $0$ is the left identity, but it doesn't say that it is also the right identity. We must therefore construct a proof that:

\begin{equation*}
h1 + 0 \stackrel{?}{≡} h1
\end{equation*}
\noindent
Luckily for us, we already have that proof defined in the \texttt{Basics.Reasoning} module as \texttt{+0}. The only problem is that our proof is in the opposite direction. It proofs $a ≡ a + 0$, not $a + 0 ≡ a$. We solve that by applying symmetry to our proof. Code for the base cases is shown in \Listing{lst:rank-proof-merge-two-pass}.

\begin{listing}[thb!]
\begin{code}
proof-1 : (l1 r1 l2 r2 : Nat) → suc ( l1 + (r1 + suc (l2 + r2)))
                              ≡ suc ((l1 + r1) + suc (l2 + r2))
proof-1 l1 r1 l2 r2 = cong suc (+assoc l1 r1 (suc (l2 + r2)))

lemma-B : (n m : Nat) → n + suc m ≡ m + suc n
lemma-B n m = trans (sym (+suc n m)) (trans (cong suc (+comm n m)) (+suc m n))

lemma-A : (a b c : Nat) → a + (b + suc c) ≡ c + suc (a + b)
lemma-A a b c = trans (+assoc a b (suc c)) (lemma-B (a + b) c)

proof-2 : (l1 r1 l2 r2 : Nat) → suc (l2 + (r2  + suc (l1 + r1)))
                              ≡ suc ((l1 + r1) + suc (l2 + r2))
proof-2 l1 r1 l2 r2 = cong suc (lemma-A l2 r2 (l1 + r1))

merge : {l r : Rank} → Heap l → Heap r → Heap (l + r)
merge empty h2 = h2 -- FIRST BASE CASE
merge {suc l} {zero} h1 empty
  = subst Heap (sym (+0 (suc l))) h1 -- SECOND BASE CASE
merge {suc .(l1-rank + r1-rank)} {suc .(l2-rank + r2-rank)}
  (node {l1-rank} {r1-rank} p1 l1≥r1 l1 r1)
  (node {l2-rank} {r2-rank} p2 l2≥r2 l2 r2)
  with p1 < p2
merge {suc .(l1-rank + r1-rank)} {suc .(l2-rank + r2-rank)}
  (node {l1-rank} {r1-rank} p1 l1≥r1 l1 r1)
  (node {l2-rank} {r2-rank} p2 l2≥r2 l2 r2)
  | true
  = subst Heap
          (proof-1 l1-rank r1-rank l2-rank r2-rank) -- FIRST RECURSIVE CASE
          (makeT p1 l1 (merge r1 (node p2 l2≥r2 l2 r2)))
merge {suc .(l1-rank + r1-rank)} {suc .(l2-rank + r2-rank)}
  (node {l1-rank} {r1-rank} p1 l1≥r1 l1 r1)
  (node {l2-rank} {r2-rank} p2 l2≥r2 l2 r2)
  | false
  = subst Heap
          (proof-2 l1-rank r1-rank l2-rank r2-rank) -- SECOND RECURSIVE CASE
          (makeT p2 l2 (merge r2 (node p1 l1≥r1 l1 r1)))
\end{code}
\caption{Implementation of merge with verified rank property.}\label{lst:rank-proof-merge-two-pass}
\end{listing}

\subsubsection{Proving merge, inductive cases}

In recursive case we know that neither $h1$ nor $h2$ is empty, ie. their sizes are given as $\suc (l1 + r1)$ and $\suc (l2 + r2)$ respectively. This means that Agda sees expected size of the merged heap as

\begin{equation*}
\suc (l1 + r1) + \suc (l2 + r2) \stackrel{+, (2)}{≡} \suc ((l1 + r1) + \suc (l2 + r2))
\end{equation*}
\noindent
This will be our goal in both proofs of recursive cases.

In the first inductive we construct the result by calling $makeT p1 l1 (merge r1 (node p2 h2-r l2 r2))$, as shown in the \Listing{lst:makeT-merge}\footnote{In the verified code shown in \Listing{lst:rank-proof-merge-two-pass} the \texttt{node} constructor takes slightly different parameters. This is because we changed the definition of \Heap datatype. This doesn't affect our reasoning here.}. Call to \texttt{node} with \texttt{l2} and \texttt{r2} as parameters produces node of rank $suc(l2 + r2)$\footnote{Remeber that \texttt{l2} denotes left child of the second tree in the code, while $l2$ denotes rank of that subtree.}. Passing it to \texttt{merge} together with \texttt{r1} gives a tree of rank $r1 + suc(l2 + r2)$ (by the type signature of \texttt{merge}). Passing result of \texttt{merge} to \texttt{makeT} produces tree of rank $suc (l1 + (r1 + suc (l2 + r2)))$ (by the type signature of \texttt{makeT}). We must therefore construct a proof that:

\begin{equation*}
\suc (l1 + (r1 + \suc (l2 + r2))) ≡ \suc ((l1 + r1) + \suc (l2 + r2))
\end{equation*}
\noindent
Appealing to congruence on $\suc$ leaves us with a proof of

\begin{equation*}
l1 + (r1 + \suc (l2 + r2)) ≡ (l1 + r1) + \suc (l2 + r2)
\end{equation*}
\noindent
Substituting $a = l1$, $b = r1$ and $c = suc (l2 + r2)$ we have

\begin{equation*}
a + (b + c) ≡ (a + b) + c
\end{equation*}
\noindent
Which is associativity of addition that we have already proved in \texttt{Basics.Reasoning}.

The proof of second inductive case is much more interesting. This time we have to prove that:

\begin{equation*}
\suc (l2 + (r2 + \suc (l1 + r1))) ≡ \suc ((l1 + r1) + \suc (l2 + r2))
\end{equation*}

Again we use congruence to deal with the outer calls to $suc$ and substitute $a = l2$, $b = r2$ and $c = l1 + r1$. This leaves us with a proof of lemma A:

\begin{equation*}
a + (b + \suc c) ≡ c + \suc (a + b)
\end{equation*}

From associativity we know that:

\begin{equation*}
a + (b + \suc c) ≡ (a + b) + \suc c
\end{equation*}

If we prove lemma B:

\begin{equation*}
(a + b) + \suc c = c + \suc (a + b)
\end{equation*}
\noindent
then we can combine it using transitivity to get the final proof. We can
rewrite lemma B as:

\begin{equation*}
n + \suc m ≡ m + \suc n
\end{equation*}
\noindent
where $n = a + b$ and $m = c$. From symmetry of \texttt{+suc} we have:

\begin{equation*}
n + (\suc m) ≡ \suc (n + m)
\end{equation*}
\noindent
Using transitivity we combine it with congruence on commutativity of addition to prove:

\begin{equation*}
\suc (n + m) ≡ \suc (m + n)
\end{equation*}
\noindent
Again, using transitivity we combine it with +suc:

\begin{equation*}
\suc (m + n) ≡ m + \suc n
\end{equation*}

Which proves lemma B and therefore the whole proof is complete. The code for this proof is shown in \Listing{lst:rank-proof-merge-two-pass}.

\subsubsection{insert}

We require that inserting an element into the heap increases itsran k by one. Now that rank is encoded as datatype index this fact is reflected in the type signatures of our functions. As previously we define insert as merge with a singleton heap. Size of singleton heap is (suc zero), while already existing heap has size n. According to definition of merge the
resulting heap will therefore have size:

\begin{equation}
(\suc \zero) + n \stackrel{+, (2)}{≡} \suc (\zero + n) \stackrel{+, (1)}{≡} \suc n
\end{equation}
\noindent
Which is size we require in the type signature of insert. This means we don't need any additional proof because expected result follows from definitions.

\subsubsection{findMin, deleteMin}

Having encoded rank at the type level we can now write total versions of \findMin and \deleteMin. By requiring that input \Heap has rank \texttt{suc n} we exclude the possibility of passing empty heap to any of these functions.

\subsection{Constructing equality proofs using transitivity}

Now that constructed two specific proofs we can focus on a more general technique used in both cases. Let's rewrite proof-2 in a different fassion to see closely what is happening at each
step. Inlining lemmas A and B into proof-2 gives:

\begin{code}
proof-2i : (l1 r1 l2 r2 : Nat) → suc (l2 + (r2  + suc (l1 + r1)))
                               ≡ suc ((l1 + r1) + suc (l2 + r2))
proof-2i l1 r1 l2 r2 =
  cong suc (trans (+assoc l2 r2 (suc (l1 + r1)))
           (trans (sym (+suc (l2 + r2) (l1 + r1)))
           (trans (cong suc (+comm (l2 + r2) (l1 + r1)))
                  (+suc (l1 + r1) (l2 + r2))))
\end{code}

We see a lot of properties combined using transitivity. In general, if we have to prove:

\begin{equation*}
a ≡ e
\end{equation*}
\noindent
and we can prove:

\begin{equation*}
a ≡ b, b ≡ c, c ≡ d, d ≡ e
\end{equation*}
\noindent
then we can combine these proofs using transitivity:

\begin{equation*}
\trans (a ≡ b) (\trans (b ≡ c) (\trans (c ≡ d) (d ≡ e)))
\end{equation*}
\noindent

While simple to use, combining proofs with transitivity can be not so obvious at first. Let's rewrite the proof we have conducted using following notation:

\begin{align*}
a& ≡[ proof 1 ]\\
b& ≡[ proof 2 ]\\
c& ≡[ proof 3 ]\\
d& ≡[ proof 4 ]\\
e& \Box
\end{align*}

Where proof 1 proves $a ≡ b$, proof 2 proves $b ≡ c$ and so on. In our
particular case this will be:

\begin{align*}
                                \suc (l2 + (r2 + \suc (l1 + r1)))& ≡[ \congr \suc ]\\
{\color{gray} \suc(} l2 + (r2 + \suc (l1 + r1))  {\color{gray})} & ≡[\assoc l2 r2 (\suc (l1 + r1))]\\
{\color{gray} \suc(} (l2 + r2) + \suc (l1 + r1)  {\color{gray})} & ≡[ \sym (\Psuc (l2 + r2) (l1 + r1))]\\
{\color{gray} \suc(} \suc ((l2 + r2) + (l1 + r1)){\color{gray})} & ≡[ \congr \suc (\comm (l2 + r2) (l1 + r1)) ]\\
{\color{gray} \suc(} \suc ((l1 + r1) + (l2 + r2)){\color{gray})} & ≡[\Psuc (l1 + r1) (l2 + r2) ]\\
{\color{gray} \suc(} (l1 + r1) + \suc (l2 + r2)  {\color{gray})} &
\end{align*}

We use grey $\suc$ to denote that everything happens under a call to \texttt{suc} (thanks to using congruence). Compare this notation with code of proof-2i in \Listing{missing-listing}. % conclusions from this comparison?

\subsection{Prooving priority property}

To prove the second property we will index nodes using Priority. Index
of value n says that "this heap can store elements with priorities
n or lower" (remember that lower priority means larger Nat). In
other words Heap indexed with 0 can store any Priority, while Heap
indexed with 3 can store priorities 3, 4 and lower, but can't store
0, 1 and 2.

As previously, Heap has two constructors:

 1) empty returns Heap n, where index n is not constrained in any
    way. This means that empty heap can be given any restriction on
    priorities of stored elements.

 2) node also creates Heap n, but this time n is constrained. If we
    store priority p in a node then:

      a) the resulting heap can only be restricted to store
         priorities at least as high as p. For example, if we
         create a node that stores priority 3 we cannot restrict
         the resulting heap to store priorities 4 and lower,
         because the fact that we store 3 in that node violates the
         restriction. This restriction is expressed by the "p ≥ n"
         parameter: if we can construct a value of type (p ≥ n)
         then existance of such a value becomes a proof that p is
         greater-equal to n. We must supply such proof to every
         node.

      b) children of a node can only be restricted to store
         priorities that are not higher than p. Example: if we
         restrict a node to store priorities 4 and lower we cannot
         restrict its children to store priorities 3 and
         higher. This restriction is expressed by index "p" in the
         subheaps passed to node constructor.

\begin{code}
data Heap : {i : Size} → Priority → Set where
  empty : {i : Size} {n : Priority} → Heap {↑ i} n
  node  : {i : Size} {n : Priority} → (p : Priority) → Rank → p ≥ n →
          Heap {i} p → Heap {i} p → Heap {↑ i} n
\end{code}

Let's demonstrate that priority invariant cannot be broken. Below
we construct a heap like this:
\begin{code}
     ?
    / \
   /   \
  E     0
\end{code}

where E means empty node and 0 means node storing Priority 0 (yes,
this heap violates the rank invariant!). We left out the root
element. The only value that can be supplied there is zero (try
giving one and see that typechecker will not accept it). This is
beacuse the value n with which 0 node can be index must obey the
restriction 0 ≥ n. This means that 0 node can only be indexed with
0. When we try to construct ? node we are thus only allowed to
insert priority 0.

\begin{code}
heap-broken : Heap zero
heap-broken = node {!!} (suc one) ge0 empty (node zero one ge0 empty empty)
\end{code}


Here is a correct heap. It stores one at the leaf and 0 at the
root. It still violates the rank invariant - we deal with that in
TwoPassMerge.RankProof.
\begin{code}
heap-correct : Heap zero
heap-correct = node zero (suc one) ge0 empty (node one one ge0 empty empty)
\end{code}


Again, we need a function to extract rank from a node
\begin{code}
rank : {b : Priority} → Heap b → Rank
rank empty            = zero
rank (node _ r _ _ _) = r
\end{code}


The first question is how to create a singleton heap, ie. a Heap
storing one element. The question we need to answer is "what
Priorities can we later store in a singleton Heap?". "Any" seems to
be a reasonable answer. This means the resulting Heap will be
indexed with zero, meaning "Priorities lower or equal to zero can
be stored in this Heap" (remember that any priority is lower or
equal to zero). This leads us to following definition:
\begin{code}
singleton : (p : Priority) → Heap zero
singleton p = node p one ge0 empty empty
\end{code}


We can imagine we would like to limit the range of priorities we
can insert into a singleton Heap. This means the resulting Heap
would be index with some b (the bound on allowed Priority
values). In such case however we are required to supply a proof
than p ≥ b. This would lead us to a definition like this:

singletonP : {b : Priority} → (p : Priority) → p ≥ b → Heap b
singletonP p p≥b = node p one p≥b empty empty

We'll return to that idea soon.

makeT now returns indexed heaps, so it requires one more parameter:
a proof that priorities stored in resulting heap are not lower than
in the subheaps.
\begin{code}
makeT : {b : Nat} → (p : Priority) → p ≥ b → Heap p → Heap p → Heap b
makeT p p≥b l r with rank l ≥ rank r
makeT p p≥b l r | true  = node p (suc (rank l + rank r)) p≥b l r
makeT p p≥b l r | false = node p (suc (rank l + rank r)) p≥b r l
\end{code}


The important change in merge is that now we don't compare node
priorities using an operator that returns Bool. We compare them
using "order" function that not only returns result of comparison,
but also supplies a proof of the result. This proof tells us
something important about the relationship between p1, p2 and
priority bound of the merged Heap. Note that we use the new proof
to reconstruct one of the heaps that is passed in recursive call to
merge. We must do this because by comparing priorities p1 and p2 we
learned something new about restriction placed on priorities in one
of the heaps and we can now be more precise in expressing these
restrictions.

Note also that despite indexing our data types with Size the
termination checker complains that merge function does not
terminate. This is not a problem in our definitions but a bug in
Agda's implementation. I leave the code in this form in hope that
this bug will be fixed in a future release of Agda. For mor details
see \url{http://code.google.com/p/agda/issues/detail?id=59\#c23}
\begin{code}
merge : {i j : Size} {p : Nat} → Heap {i} p → Heap {j} p → Heap p
merge empty h2 = h2
merge h1 empty = h1
merge .{↑ i} .{↑ j}
  (node {i} p1 l-rank p1≥p l1 r1)
  (node {j} p2 r-rank p2≥p l2 r2)
  with order p1 p2
merge .{↑ i} .{↑ j}
  (node {i} p1 l-rank p1≥p l1 r1)
  (node {j} p2 r-rank p2≥p l2 r2)
  | le p1≤p2
  = makeT p1 p1≥p l1 (merge {i} {↑ j} r1 (node p2 r-rank p1≤p2 l2 r2))
merge .{↑ i} .{↑ j}
  (node {i} p1 l-rank p1≥p l1 r1)
  (node {j} p2 r-rank p2≥p l2 r2)
  | ge p1≥p2
  = makeT p2 p2≥p l2 (merge {↑ i} {j} (node p1 l-rank p1≥p2 l1 r1) r2)
\end{code}


When writing indexed insert function we once again have to answer a
question of how to index input and output Heap. The easiest
solution is to be liberal: let us require that the input and output
Heap have no limitations on Priorities they can store. In other
words, let they be indexed by zero:
\begin{code}
insert : Priority → Heap zero → Heap zero
insert p h = merge (singleton p) h
\end{code}


But what if we actaully want to enforce bounds imposed on the heap
by its index? In that situation we are required to provide proof
that the priority we are inserting into the Heap can really be
insterted into it. To do this we will need a few additional
functions. First of all we need a new singleton function that will
construct a singleton Heap with a bound. This requires us to supply
additional parameter that is evidence that priority we just
inserted into our singleton heap is lower than the bound.
\begin{code}
singletonE : {b : Priority} → (p : Priority) → p ≥ b → Heap b
singletonE p p≥b = node p one p≥b empty empty
\end{code}


We also need a proof of transitivity of ≥. We proceed by induction
on c. Our base case is:

  a ≥ b ∧ b ≥ 0 ⇒ a ≥ 0

In other words if c is 0 then ge0 proofs the property. If c is not
zero, then b is also not zero (by definitions of data constructors
of ≥) and hence a is also not zero. This gives us second equation
that is a recursive proof on ≥trans.
\begin{code}
≥trans : {a b c : Nat} → a ≥ b → b ≥ c → a ≥ c
≥trans a≥b        ge0      = ge0
≥trans (geS a≥b) (geS b≥c) = geS (≥trans a≥b b≥c)
\end{code}


Having proved the transitivity of ≥ we can construct a function
that loosens the bound we put on a heap. If we have a heap with a
bound p - meaning that all priorities in a heap are guaranteed to
be lower than or equal to p - and we also have evidence than n is a
priority higher than p then we can change the restriction on the
heap so that it accepts higher priorites. For example if we have
Heap 5, ie. all elements in the heap are 5 or greater, and we have
evidence that 5 ≥ 3, then we can convert Heap 5 to Heap 3. Note how
we are prevented from writing wrong code: if we have Heap 3 we
cannot convert it to Heap 5. This is not possible from theoretical
point of viwe: Heap 3 may contain 4, but Heap 5 is expected to
contain elements not smaller than 5. It is also not possible
practically: thanks to our definition of ≥ we can't orivide
evidence that 3 ≥ 5 because we cannot construct value of that type.
\begin{code}
liftBound : {p b : Priority} → b ≥ p → Heap b → Heap p
liftBound b≥n empty = empty
liftBound b≥n (node p rank p≥b l r)
  = node p rank (≥trans p≥b b≥n) l r
\end{code}


With singletonE and liftBound we can construct insert function that
allows to insert element with priority p into a Heap bound by b,
but only if we can supply evidence that p ≥ b, ie. that p can
actually be stored in the heap.
\begin{code}
insertE : {b : Priority} → (p : Priority) → p ≥ b → Heap p → Heap b
insertE p p≥b h = merge (singletonE p p≥b) (liftBound p≥b h)
\end{code}


Again, findMin and deletMin are incomplete
\begin{code}
findMin : {b : Priority} → Heap b → Priority
findMin empty            = {!!}
findMin (node p _ _ _ _) = p
\end{code}


deleteMin requires a bit more work than previously. First, notice
that the bound placed on the input and output heaps is the
same. This happens because relation between priorities in a node
and it's children is ≥, not > (ie. equality is allowed). We cannot
therefore guearantee that bound on priority will increase after
removing highest-priority element from the Heap. When we merge
left and right subtrees we need to lift their bounds so they are
now both b (not p).
\begin{code}
deleteMin : {b : Priority} → Heap b → Heap b
deleteMin empty                 = {!!}
deleteMin (node p rank p≥b l r) = merge (liftBound p≥b l) (liftBound p≥b r)
\end{code}
