\section{Weight biased leftist heaps}\label{sec:wblh}

A heap is a tree-based data structure used to implement priority queue. Each node in the heap satisfies \textit{priority property}: priority of element at the node is not lower than priority of the children nodes\footnote{I will also refer to children of a node as ``subtrees'' or ``subheaps''.}. Element with the highest priority is stored at the root. Access to it has O(1) complexity.

Weight biased leftist tree \cite{ChoSah96} is a binary tree that satisfies \textit{rank property}: for each node rank of its left child is not smaller than rank of its right child. Rank of a tree is defined as its size (number of nodes). Weight biased leftist tree that satisfies priority property is called a weight biased leftist heap (WBLH).

\textit{Right spine} of a node is the rightmost path from that node to an empty node. From priority property it follows that right spine of a WBLH is an ordered list\footnote{In fact, any path from root to a leaf is!}. Two weight biased leftist heaps can be merged in O(log n) time by merging their right spines in the same way one merges ordered lists and then swapping children along the right spine of merged heap to restore rank property \cite{Oka99}. Inserting new element into WBLH can be defined as merging existing heap with a newly created singleton heap. Deleting element with the highest priority can be defined as merging children of the root element.
