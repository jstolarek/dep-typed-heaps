\section{Introduction}

Formal verification is a subject that constantly attracts a great deal of attention from the research community. Static type systems are considered to be a lightweight verification method but they can be very powerful and precise as well. Dependent type systems in languages like Agda~\cite{Nor07}, Idris~\cite{Bra13} or Coq~\cite{coq} have been succesfully applied in many practical verification tasks. But verification techniques enabled by dependent types are not yet as widely used as they could potentially be. This paper contributes to changing that.

\subsection{Motivation}

Two things motivted me to write this paper. Firstly, while there are many tutorials on dependently typed programming and basics of verification, I could find little material demonstrating how to put verification to practical use. A must-read introductory level paper ``Why Dependent Types Matter'' by Altenkirch, McKinna and McBride \cite{AltMcBMcK05} that demonstrates how to use power of dependent types to prove correctness of merge sort algorithm actually elides many proof details that are required in a real-world application. I wanted to fill in that missing gap and write a tutorial that picks up where other tutorials have ended. My second motivation came from reading Okasaki's classical ``Purely Functional Data Structures''~\cite{Oka99}. Despite book's title many presented implementations are not purely functional as they make use of impure exceptions to handle corner cases (eg. taking head from an empty list). I realized that using dependent types allows to do better and it will be instructive to build a provably correct purely functional data structure on top of Okasaki's presentation. In the end this paper is not only a tutorial but also a case study of a weight biased leftist heap implemented in a dependently typed setting. My goal is to teach the reader how operations on a data structure can be proved correct by constructing their proof from elementary components.

\subsection{Companion code}

This tutorial comes with a standalone companion code written in Agda 2.3.4\footnote{\url{http://ics.p.lodz.pl/~stolarek/_media/pl:research:dep-typed-wbl-heaps.tar.gz}}. I assume the reader is reading companion code along with the paper. Due to space limitations I elide some proofs that are detailed in the code using Notes convention addapted from GHC project~\cite{MarPey12}.

``Living'' version of companion code is available at GitHub\footnote{\url{https://github.com/jstolarek/dep-typed-wbl-heaps}} and it may receive updates after the paper has been published.

\subsection{Assumptions}

I assume that reader has basic understanding of Agda, some elementary definitions and proofs. In particular I assume that reader is familiar with definition of natural numbers (\texttt{Nat}s) and their addition (\texttt{+}) as well as proofs of basic properties of addition like associativity, commutativity or 0 as right identity ($a + 0 ≡ a$). Reader should also understand \texttt{refl} with its basic properties (symmetry, congruence, transitivity and substitution), know the concept of ``data as evidence'' and other ideas presented in ``Why Dependent Types Matter'' \cite{AltMcBMcK05} as we will build upon them. All of these are implemented in the \texttt{Basics} module in the companion code. Module \texttt{Basics.Reasoning} reviews in detail the above-mentioned proofs. If you are not familiar with some of the concepts above I recommend taking a look at tutorial papers listed on Agda Wiki\footnote{\url{http://wiki.portal.chalmers.se/agda/}}.

\subsection{Notation and conventions}

In the rest of the paper I use \texttt{typewriter font} to denote a heap and \textit{italic type} to denote its rank. The description of merge algorithm will mention heaps \texttt{h1} and \texttt{h2} with ranks \textit{h1} and \textit{h2}, respectively, their left children (\texttt{l1} in \texttt{h1} and \texttt{l2} in \texttt{h2}) and right children (\texttt{r1} in \texttt{h1} and \texttt{r2} in \texttt{h2}). \texttt{p1} and \texttt{p2} are the priorities of root elements in \texttt{h1} and \texttt{h2}, respectively. In the text I will use $\oplus$ to denote heap merging operation. So \texttt{h1}$\oplus$\texttt{h2} is a heap created by merging \texttt{h1} with \texttt{h2}, while \textit{h1}$\oplus$\textit{h2} is the rank of merged heap.

In the text I will use numerals to represent \texttt{Nat}s, although in the code I use encoding based on \texttt{zero} and \texttt{suc}. Thus 2 in the text corresponds to \texttt{suc (suc zero)} in the source code.

I use \texttt{\hilight{\{ \}?}} in code listings to represent Agda holes.

Remember that any sequence of Unicode characters is a valid identifier in Agda. Thus \texttt{l≥r} is an identifier, while \texttt{l ≥ r} is application of \texttt{≥} operator to \texttt{l} and \texttt{r} operands.

\subsection{Contributions}

This paper contributes the following:

\begin{itemize}
 \item Section~\ref{sec:no-proofs} presents the problem of partiality of functions operating on a weight biased leftist heap. While the problem in general is well-known the solution to this particular case can be combined with verification of data structure's invariants. This is done in Section~\ref{sec:rank-property}.
 \item Section~\ref{sec:eq-proofs-using-trans} outlines a technique for constructing equality proofs using transitivity of propositional equality. This simple, standalone technique provides ground for understanding verification mechanisms provided by Agda's standard library.
 \item Section~\ref{sec:single-pass-merge-proof-by-comp} uses the technique introduced in Section~\ref{sec:eq-proofs-using-trans} to prove code obtained by inlining one function into another. This shows how programs created from smalls, verified components can be proven correct by composing existing proofs.
 \item Section~\ref{sec:priority-invariant} contains a case study of how a proof of a data structure's invariant influences the design of that structure's API.
\end{itemize}
