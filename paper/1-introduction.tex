\section{Introduction}
% Purpose of this paper is to present formal verification of a known purely functional data structure (well known).

\subsection{Companion code}

This tutorial comes with companion code available at my web page \cite{js-webhome}\footnote{``Living'' version of companion code is available at GitHub \cite{js-github}.}. Code is written in Agda 2.3.3, which is a development version of the compiler as of January 2014. See Agda Wiki \cite{agda-wiki} for instructions how to install development version of Agda. I assume that the reader is reading the companion code along with the paper. Due to space limitations I elide some proofs that are described in detail in the companion code.

\subsection{Assumptions}

I assume that reader has basic understanding of Agda and using dependent types to conduct proofs. In particular I assume that reader is familiar with definition of natural numbers (\texttt{Nat}s) and their addition (\texttt{+}) as well as proofs of basic properties of addition like associativity, commutativity or 0 as right identity ($a + 0 ≡ a$). Reader should also understand \texttt{refl} with its basic properties (symmetry, congruence, transitivity and substitution), know the concept of ``data as evidence'' and other ideas presented in ``Why Dependent Types Matter'' \cite{AltMcBMcK05} as we will build upon them. If you are not familiar with some of the concepts listed above I recommend taking a look at tutorial papers listed on Agda Wiki \cite{agda-wiki}.

\subsection{Notation}

In the rest of the paper I use \texttt{typewriter font} to denote a heap and \textit{italic type} to denote its rank. The description of merge algorithm will mention heaps \texttt{h1} and \texttt{h2} with ranks \textit{h1} and \textit{h2}, respectively, their left children (\texttt{l1} in \texttt{h1} and \texttt{l2} in \texttt{h2}) and right children (\texttt{r1} in \texttt{h1} and \texttt{r2} in \texttt{h2}). \texttt{p1} and \texttt{p2} are the priorities of root elements in \texttt{h1} and \texttt{h2}, respectively. I will use $\oplus$ to denote heap merging operation. So \texttt{h1}$\oplus$\texttt{h2} is a heap created by merging \texttt{h1} with \texttt{h2}, while \textit{h1}$\oplus$\textit{h2} is the rank of merged heap.

In the text I will use numerals to represent \texttt{Nat}s, although in code I use encoding based on \texttt{zero} and \texttt{suc}. Thus 2 in the text corresponds to \texttt{suc (suc zero)} in the source code.

\subsection{Contributions}

This paper contains following contributions:

\begin{itemize}
 \item % explains how to build equality proofs using transitivity, allows this basic thing allows to understand more complex stuff (standard library).
 \item % how to construct proofs from smaller proofs when code is transformed
 \item % picking up were other tutorials left - in intro
 \item % demonstrating how to verify a well known functional data structure using techniques developed earlier - in intro
\end{itemize}
